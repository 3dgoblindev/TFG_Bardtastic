% !TeX root = ../tfg.tex
% !TeX encoding = utf8

\chapter{Conclusiones y trabajo a futuro}\label{conclusiones}

\begin{flushright}
	\itshape
	``Legacy. What is a legacy?\\
	It’s planting seeds in a garden you never get to see.\\
	I wrote some notes at the beginning of a song someone will sing for me.’’\\[1ex]
	\hrulefill\\[-0.5ex]
	\small Lin-Manuel Miranda, \textit{Hamilton} (2015) \cite{Hamilton2015}
\end{flushright}



\vspace{2em}


El presente Trabajo de Fin de Grado ha tenido como objetivo el diseño e implementación de un prototipo funcional del videojuego \textit{Bardtastic}, un \textit{deckbuilder} narrativo en el se encarna a un bardo que debe entretener a una audiencia mediante el uso estratégico de cartas que representan elementos narrativos y puestas en escena.  
El proyecto ha combinado la faceta creativa del diseño de juegos con el rigor técnico de la Ingeniería Informática y los fundamentos matemáticos de la Teoría de Juegos y la Inteligencia Artificial.

\section{Síntesis y logros del proyecto}

A lo largo del desarrollo se ha abordado la construcción de un prototipo completo que incluye la lógica de juego, la gestión de cartas, la interfaz de usuario y un agente rival controlado por IA.  
El prototipo cumple plenamente su función como \textbf{prueba de concepto}: valida la idea jugable y demuestra la viabilidad de un sistema narrativo competitivo basado en reglas probabilísticas y toma de decisiones racional.

En el plano teórico, se han aplicado formalismos propios de la Teoría de Juegos, concretamente los modelos de juegos de suma cero y los algoritmos de exploración de árboles de decisión.  
El agente implementado utiliza una exploración de árboles con una función heurística, ajustada a la naturaleza aleatoria y a la información oculta del juego, y ha mostrado un comportamiento \textbf{coherente, racional y competitivo}, capaz de evaluar las jugadas del jugador humano y responder con decisiones óptimas dentro de la información disponible.

Desde el punto de vista de la ingenieria, se ha puesto especial énfasis en la \textbf{arquitectura modular y extensible del \textit{software}}.  
El diseño basado en clases bien encapsuladas, el uso de componentes reutilizables y la separación entre la lógica del juego, la IA y la presentación visual, permiten que el sistema pueda ampliarse fácilmente con nuevos elementos: cartas, mecánicas, tipos de audiencia o incluso agentes con distintos estilos de juego.  
Este principio de modularidad garantiza la mantenibilidad del código y su posible reutilización en versiones futuras o proyectos derivados.

\section{Valoración de los objetivos}

Los objetivos establecidos al inicio del TFG se han alcanzado satisfactoriamente:

\begin{itemize}
  \item Se ha diseñado e implementado un videojuego tipo \textit{deckbuilder} con una propuesta original y temática coherente.
  \item Se ha aplicado la Teoría de Juegos para modelar la toma de decisiones de los oponentes.
  \item Se ha desarrollado un agente de IA funcional, basado en exploración de árboles, capaz de jugar de forma racional contra el usuario.
  \item Se ha validado experimentalmente el concepto de juego, confirmando la viabilidad de su mecánica central y del modelo de atención del público.
\end{itemize}

Estos resultados integran los conocimientos adquiridos en el grado, demostrando la capacidad de aplicar conceptos teóricos a un desarrollo \textit{software} real y funcional. Además, se amplían y combinan hacia ámbitos como el diseño.

\section{Trabajo a futuro}

El proyecto deja abiertas varias líneas de desarrollo que pueden explorarse en trabajos posteriores:

\begin{itemize}
  \item \textbf{Ampliación del contenido jugable:} incorporación de más cartas, efectos y sinergias que profundicen en la estrategia del jugador y aumenten la rejugabilidad.
  \item \textbf{Evolución de la IA:} sustitución del algoritmo determinista por aproximaciones estocásticas basadas en \textit{Monte Carlo Tree Search (MCTS)}, capaces de manejar incertidumbre y explorar grandes espacios de decisión.
  \item \textbf{Optimización del rendimiento:} análisis de costes computacionales de la exploración de árboles y paralelización de cálculos en GPU o hilos independientes.
  \item \textbf{Integración narrativa y visual:} mejora el diseño de personajes, arte y sonido para convertir el prototipo en una obra más llamativa.
  \item \textbf{Comercialización:} pasar de prototipo a juego completo y viable para lanzar en el mercado de PC y consolas.
\end{itemize}

En conjunto, \textit{Bardtastic} demuestra que un videojuego puede servir como banco de pruebas para conceptos avanzados de decisión racional y equilibrio estratégico.  
El enfoque modular adoptado y la base teórica empleada aseguran que el proyecto pueda crecer, diversificarse y evolucionar tanto en el terreno académico como en el creativo.

\endinput
%--------------------------------------------------------------------
% FIN DEL CAPÍTULO.
%--------------------------------------------------------------------
