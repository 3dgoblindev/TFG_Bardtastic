% !TeX root = ../tfg.tex
% !TeX encoding = utf8

\chapter{Conclusiones y trabajo a futuro}

El presente Trabajo de Fin de Grado ha tenido como objetivo el diseño e implementación de un prototipo funcional del videojuego \textit{Bardtastic}, un \textit{deckbuilder} narrativo en el que el jugador encarna a un bardo que debe entretener a una audiencia mediante el uso estratégico de cartas que elementos narrativos y puestas en escena.  
El proyecto ha combinado la faceta creativa del diseño de juegos con el rigor técnico de la Ingeniería Informática y los fundamentos matemáticos de la Teoría de Juegos y la Inteligencia Artificial.

\section{Síntesis y logros del proyecto}

A lo largo del desarrollo se ha abordado la construcción de un prototipo completo que incluye la lógica de juego, la gestión de cartas, la interfaz de usuario y un agente rival controlado por IA.  
El prototipo cumple plenamente su función como \textbf{prueba de concepto}: valida la idea jugable y demuestra la viabilidad de un sistema narrativo competitivo basado en reglas probabilísticas y toma de decisiones racional.

En el plano teórico, se han aplicado formalismos propios de la Teoría de Juegos, concretamente los modelos de juegos de suma cero y los algoritmos de exploración de árboles de decisión.  
El agente implementado utiliza una exploración de árboles con una función heurística, ajustada a la naturaleza aleatoria y a la información oculta del juego, y ha mostrado un comportamiento \textbf{coherente, racional y competitivo}, capaz de evaluar las jugadas del jugador humano y responder con decisiones óptimas dentro de la información disponible.

Desde el punto de vista de la ingenieria, se ha puesto especial énfasis en la \textbf{arquitectura modular y extensible del software}.  
El diseño basado en clases bien encapsuladas, el uso de componentes reutilizables y la separación entre la lógica del juego, la IA y la presentación visual, permiten que el sistema pueda ampliarse fácilmente con nuevos elementos: cartas, mecánicas, tipos de audiencia o incluso agentes con distintos estilos de juego.  
Este principio de modularidad garantiza la mantenibilidad del código y su posible reutilización en versiones futuras o proyectos derivados.

\section{Valoración de los objetivos}

Los objetivos planteados al inicio del TFG, tanto la fundamentación matemática como el desarrollo del prototipo y su agente, se han cumplido de manera satisfactoria, alcanzando los resultados previstos en ambas líneas del proyecto.

En relación con el \textbf{objetivo general 1} (fundamentación matemática y teórica), se ha construido un marco formal completo basado en la teoría de juegos que permite describir rigurosamente qué es un juego, cuáles son sus estrategias y cómo se caracterizan sus equilibrios. Este análisis ha permitido clasificar el enfrentamiento central de \textit{Bardtastic}, justificando desde criterios matemáticos las elecciones de diseño y motivando la estructura de decisión del agente rival. Asimismo, se han presentado los conceptos esenciales (equilibrio de Nash, estrategias dominadas, modelos estáticos y dinámicos, información completa e incompleta) situando el videojuego dentro de un marco teórico sólido.

En cuanto al \textbf{objetivo general 2} (diseño e implementación del prototipo y del agente), se ha desarrollado un videojuego funcional con mecánicas originales basadas en contar historias, rimas y estrategia de cartas. La arquitectura implementada en Unreal Engine separa adecuadamente la lógica del juego, la interfaz y la inteligencia artificial, lo que facilita la extensibilidad del sistema. El agente rival, basado en exploración mediante \emph{Depth-First Search} y guiado por una función de utilidad heurística, ha demostrado un comportamiento racional y coherente con las limitaciones del entorno, ofreciendo un desafío creíble al jugador. Finalmente, las pruebas funcionales y el \textit{playtesting} supervisado han servido para ajustar tanto las mecánicas como el comportamiento del agente, validando la jugabilidad y la claridad del sistema de atención del público.

En conjunto, el trabajo ha logrado integrar de forma efectiva la modelización matemática con el diseño y la ingeniería del \textit{software}, resultando en un prototipo jugable que demuestra la aplicabilidad de la teoría en un contexto creativo y técnico real.


\section{Trabajo futuro}

El proyecto deja abiertas varias líneas de desarrollo que pueden explorarse en trabajos posteriores:

\begin{itemize}
  \item \textbf{Ampliación del contenido jugable:} incorporación de más cartas, efectos y sinergias que profundicen en la estrategia del jugador y aumenten la rejugabilidad.
  \item \textbf{Evolución de la IA:} sustitución del algoritmo determinista por aproximaciones estocásticas basadas en \textit{Monte Carlo Tree Search (MCTS)}, capaces de manejar incertidumbre y explorar grandes espacios de decisión.
  \item \textbf{Optimización del rendimiento:} análisis de costes computacionales de la exploración de árboles y paralelización de cálculos en GPU o hilos independientes.
  \item \textbf{Integración narrativa y visual:} mejora el diseño de personajes, arte y sonido para convertir el prototipo en una obra más llamativa.
  \item \textbf{Comercialización:} pasar de prototipo a juego completo y viable para lanzar en el mercado de PC y consolas.
\end{itemize}

En conjunto, \textit{Bardtastic} demuestra que un videojuego puede servir como banco de pruebas para conceptos avanzados de decisión racional y equilibrio estratégico.  
El enfoque modular adoptado y la base teórica empleada aseguran que el proyecto pueda crecer, diversificarse y evolucionar tanto en el terreno académico como en el creativo.

\endinput
%--------------------------------------------------------------------
% FIN DEL CAPÍTULO.
%--------------------------------------------------------------------
