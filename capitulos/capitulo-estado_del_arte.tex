% !TeX root = ../tfg.tex
% !TeX encoding = utf8
\chapter{Estado del arte}

%REVISAR SOCORRO PORFA Y METE FOTOS

El presente capítulo examina el panorama actual de los videojuegos que comparten fundamentos conceptuales o mecánicos con \textit{Bardtastic}. El objetivo es identificar los títulos más relevantes dentro del género \textit{deckbuilder} y de los juegos con mecánicas narrativas, analizar sus aportaciones y limitaciones, y situar a \textit{Bardtastic} en relación con ellos.  

El análisis se estructura en dos apartados complementarios. En primer lugar, se estudia el contexto de diseño y la evolución del género \textit{roguelike deckbuilder}, comparando \textit{Bardtastic} con obras representativas como \textit{Slay the Spire}, \textit{Balatro} y \textit{Storyteller}. En segundo lugar, se examinan los referentes artísticos y temáticos que inspiran su identidad visual y tono humorístico.

\section{Diseño y género}

\subsection{Origen y evolución del \textit{deckbuilding roguelike}}

El concepto de \textit{deckbuilding} (la construcción de un mazo de cartas como núcleo de la experiencia jugable) tiene su origen en los juegos de mesa. El punto de partida más reconocido es \textit{Dominion} (Vaccarino, 2008), que introdujo la idea de que los jugadores comienzan con un conjunto de cartas básicas y, durante la partida, adquieren nuevas cartas que definen su estrategia \cite{Dominion2008}.  
Esta mecánica trasladó la gestión y optimización de recursos al nivel de la propia estructura del mazo, generando una capa de planificación y rejugabilidad inédita en los juegos de cartas tradicionales.

Posteriormente, títulos como \textit{Ascension} (Gary Games, 2010) \cite{Ascension2010} consolidaron y popularizaron el género, incorporando variaciones en la dinámica de adquisición y uso de cartas. A partir de ahí, surgieron múltiples reinterpretaciones que ampliaron el concepto, como \textit{Quarriors!} (Elliott y Kovalic, 2011) \cite{Quarriors2011} o \textit{Marvel Dice Masters} (Elliott y Wilson, 2014) \cite{MarvelDiceMasters2014}, donde la construcción se centra en conjuntos de dados o fichas en lugar de cartas.

Por su parte, el término \textit{roguelike} proviene del clásico \textit{Rogue} (Toy, Wichman y Arnold, 1980) \cite{Rogue1980}, caracterizado por la generación procedural de niveles, la \textit{permadeath} (al morir en una partida, se pierde todo el progreso) y la estructura de partidas autoconclusivas. Esta filosofía de diseño, basada en la repetición significativa y la toma de decisiones bajo incertidumbre, fue heredada por títulos contemporáneos como \textit{The Binding of Isaac} (McMillen, 2011) \cite{BindingOfIsaac2011} o \textit{Hades} (Supergiant Games, 2020) \cite{Hades2020}, que consolidaron el modelo de progresión mediante \textit{runs} independientes y mejora acumulativa del jugador.

La convergencia entre ambos géneros culminó con \textit{Slay the Spire} (MegaCrit, 2019) \cite{SlayTheSpire2019}, que estableció un nuevo estándar al combinar la construcción progresiva de mazos con la estructura de juego \textit{roguelike}. Este título demostró que las decisiones estratégicas propias del \textit{deckbuilding} podían integrarse eficazmente en un ciclo de exploración y riesgo característico de los \textit{roguelikes}, generando un alto valor de rejugabilidad y profundidad táctica.

En años recientes, el género ha continuado diversificándose. Obras como \textit{Balatro} (LocalThunk, 2024) \cite{Balatro2024} o \textit{Dungeon Clawler} (Stray Fawn Studio, 2023) \cite{DungeonClawler2023} han reinterpretado la fórmula introduciendo estéticas más ligeras, sistemas de puntuación o capas narrativas simplificadas.  

En este contexto, \textit{Bardtastic} se presenta como una evolución natural del género, que introduce a su vez un enfoque innovador: mantiene la estructura estratégica y la progresión propias del \textit{deckbuilder roguelike}, pero desplaza el foco hacia la creatividad narrativa, el humor y la riqueza temática.


\subsection{\textit{Slay the Spire}}

\begin{figure}[H]
    \centering
    \includegraphics[width=1\textwidth]{img/slay_the_spire.jpg}
    \caption{Captura de pantalla de \textit{Slay the Spire} \cite{SlayTheSpire2019}.}
    \label{fig:slay_img}
\end{figure}

\textit{Slay the Spire} es considerado el referente contemporáneo del género \textit{roguelike deckbuilder}. Su estructura combina la progresión a través de combates por turnos con la construcción dinámica de un mazo que evoluciona a medida que el jugador avanza. La rejugabilidad proviene de la aleatoriedad en las rutas, enemigos y cartas disponibles, lo que exige una gestión estratégica del riesgo y de los recursos por parte del jugador \cite{SlayTheSpire2019}.

\textbf{Relación con \textit{Bardtastic}.}  
Ambos títulos comparten la naturaleza \textit{roguelike deckbuilder} y un núcleo de decisión basado en la optimización del mazo. Además, ambos presentan la estructura de un viaje en el que el jugador puede modificar la ruta a seguir, equilibrando así la relación entre riesgo y recompensa.  
Sin embargo, mientras que \textit{Slay the Spire} traduce cada carta en acciones de combate, \textit{Bardtastic} las convierte en fragmentos narrativos y recursos expresivos. Por otra parte, la carga narrativa en \textit{Bardtastic} no se limita a los enfrentamientos principales, sino que se expande a través de una trama más amplia, poblada de personajes y situaciones que evolucionan a lo largo de la partida. En contraste, la propuesta narrativa de \textit{Slay the Spire} resulta más lineal y funcional, centrada únicamente en el ascenso a una torre habitada por monstruos y cultistas.

\textbf{Limitaciones detectadas.}  
\begin{itemize}
    \item Ausencia de un componente narrativo emergente.
    \item Repetitividad temática: las acciones se articulan casi exclusivamente en torno al combate.
    \item Falta de un tono humorístico o identitario que trascienda la fantasía genérica.
\end{itemize}

\subsection{\textit{Balatro}}

\begin{figure}[H]
    \centering
    \includegraphics[width=1\textwidth]{img/balatro.jpg}
    \caption{Captura de pantalla de \textit{\textbf{Balatro}} \cite{Balatro2024}.}
    \label{fig:balatro_img}
\end{figure}

\textit{Balatro} expande el concepto de \textit{deckbuilder} aplicándolo a las combinaciones del póker, mediante un sistema basado en la puntuación y en la optimización de sinergias entre cartas. Su éxito radica en la accesibilidad, el ritmo ágil de las partidas y el carácter adictivo derivado de la mejora incremental del mazo \cite{Balatro2024}.

\textbf{Relación con \textit{Bardtastic}.}  
Ambos títulos comparten un enfoque centrado en el uso expresivo de las cartas como un sistema de reglas autosuficiente, así como una estética desenfadada que atenúa la carga estratégica. No obstante, \textit{Balatro} se orienta casi por completo hacia la optimización numérica, mientras que \textit{Bardtastic} traslada esa estructura al terreno de la interpretación, la narrativa y la creatividad.  
Por otra parte, \textit{Balatro} carece de cualquier tipo de narrativa o representación diegética del jugador, mientras que \textit{Bardtastic} incorpora una historia y un elenco de personajes que contextualizan las acciones y refuerzan la dimensión expresiva de la experiencia.

\textbf{Limitaciones detectadas.}  
\begin{itemize}
    \item Ausencia de progresión narrativa o contexto temático.
    \item Falta de interacción directa con oponentes.
    \item Enfoque puramente abstracto y numérico, sin dimensión narrativa o temática.
\end{itemize}

\subsection{\textit{Storyteller}}

\begin{figure}[H]
    \centering
    \includegraphics[width=1\textwidth]{img/Storyteller.jpg}
    \caption{Captura de pantalla de \textit{\textbf{Storyteller}} \cite{Storyteller2023}.}
    \label{fig:storyteller_img}
\end{figure}

\textit{Storyteller} propone un enfoque radicalmente distinto: el jugador compone historias a partir de módulos predefinidos (personajes, escenarios y acciones) con el objetivo de cumplir determinadas condiciones narrativas \cite{Storyteller2023}. En este caso, no se trata de un \textit{roguelike deckbuilder}, sino de un juego de puzles cuyo valor reside en el uso de la narrativa como mecánica central, invitando a la experimentación dentro de un sistema cerrado de combinaciones posibles.

\textbf{Relación con \textit{Bardtastic}.}  
Ambos juegos sitúan la narración en el núcleo de la experiencia jugable. Sin embargo, mientras \textit{Storyteller} se estructura como una sucesión de desafíos con soluciones predefinidas, \textit{Bardtastic} propone una narrativa competitiva y emergente, donde la historia surge de la combinación libre de cartas y de la interacción entre los jugadores.

\textbf{Limitaciones detectadas.}  
\begin{itemize}
    \item Escasa libertad creativa más allá de las soluciones previstas.
    \item Ausencia de una dimensión estratégica o de construcción de mazos.
    \item Falta de un sistema competitivo o de valoración del relato.
\end{itemize}


\section{Síntesis comparativa}

La Tabla~\ref{tab:comparativa_estado_arte} resume las principales características de los títulos analizados en relación con \textit{Bardtastic}.

\begin{table}[H]
\centering
\begin{tabular}{lcccc}
\toprule
\textbf{Característica} & \textbf{Slay the Spire} & \textbf{Balatro} & \textbf{Storyteller} & \textbf{Bardtastic} \\
\midrule
\textit{Deckbuilding} & Sí & Sí & No & Sí \\
\textit{Mecánicas Narrativas} & No & No & Parcial & Sí \\
\textit{Competitividad directa} & No & No & No & Sí \\
\textit{Tono cómico} & No & Parcial & Sí & Sí \\
\textit{Creatividad del jugador} & Media & Baja & Alta & Alta \\
\textit{Narrativa global} & Parcial & No & No & Sí \\
\bottomrule
\end{tabular}
\caption{Comparativa de características entre obras de referencia y \textit{Bardtastic}.}
\label{tab:comparativa_estado_arte}
\end{table}

\subsection{Conclusiones}

El análisis evidencia que los títulos existentes tienden a concentrarse en uno de los tres ejes principales: la estrategia del mazo (\textit{Slay the Spire}, \textit{Balatro}) o la construcción narrativa (\textit{Storyteller}). Ninguno integra ambos de manera orgánica ni introduce una dimensión competitiva centrada en la interpretación y el humor.

\textit{Bardtastic} se diferencia al situar la creación narrativa como núcleo mecánico dentro de una estructura \textit{deckbuilder roguelike}, dotando a cada carta de una función expresiva y temática. El resultado es un sistema que combina estrategia y creatividad, invitando al jugador no solo a optimizar su mazo, sino a construir relatos coherentes, ingeniosos y cómicos en un contexto competitivo.  

Esta síntesis entre planificación estratégica, libertad narrativa y tono humorístico posiciona a \textit{Bardtastic} como una propuesta singular dentro del panorama actual, ampliando el potencial expresivo del \textit{deckbuilding} más allá de la mera optimización mecánica.

\section{Artístico y temático}
%pongo capturas de todo y digo que es el moodboard

El apartado artístico de \textit{Bardtastic} se nutre de referentes que combinan fantasía, humor y absurdo desde una perspectiva visual. Títulos como \textit{Munchkin} (Steve Jackson Games, 2001) \cite{Munchkin2001}y series como \textit{Adventure Time} (Pendleton Ward, 2010) \cite{AdventureTime2010} constituyen influencias directas en el tono y la dirección artística del proyecto. Ambos ejemplos se caracterizan por una estética caricaturesca y colorida que, lejos de trivializar la experiencia, refuerza la dimensión cómica y accesible del universo que representan.

Otra de las principales referencias estéticas de \textit{Bardtastic} es la corriente arquitectónica y decorativa desarrollada por el Grupo Memphis. Este movimiento se distingue por el uso de formas geométricas audaces, colores vibrantes y patrones de alto contraste, en una búsqueda deliberada de ruptura con la sobriedad del diseño modernista \cite{MemphisDesign}.  
En el contexto del juego, se busca integrar esta filosofía visual con los motivos de la fantasía medieval, generando una identidad nueva y reconocible que combine lo lúdico con lo excéntrico.

Esta elección estética contrasta con el tono sombrío y serio predominante en muchos \textit{deckbuilders roguelike} contemporáneos, proponiendo en su lugar una atmósfera más ligera y desenfadada. La fantasía de \textit{Bardtastic} no aspira a la épica heroica oscura, sino a la improvisación, el ingenio y la exageración, en coherencia con su temática de bardo y su énfasis en la creatividad narrativa.

El resultado es una identidad visual y tonal que refuerza la propuesta mecánica del título: mientras el jugador construye historias, el mundo que las enmarca mantiene un tono humorístico, irreverente y expresivo, fomentando una experiencia coherente entre arte, jugabilidad y mensaje.


\endinput

%--------------------------------------------------------------------
% FIN DEL CAPÍTULO. 
%--------------------------------------------------------------------
